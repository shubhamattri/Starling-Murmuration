\documentclass[a4paper]{article}

\usepackage[english]{babel}
\usepackage[utf8]{inputenc}
\usepackage{algorithm,algpseudocode}
\usepackage{amsmath}
\usepackage{graphicx}
\usepackage[colorinlistoftodos]{todonotes}

\title{\textbf{Mathematical Modeling of Starling Murmuration}}

\author{Shubham (2016CS10371) \\ Sanket Hire (2016CS50402)}

\date{\today}

\begin{document}
\maketitle

\begin{abstract}
Craig Reynold in 1986 was the first to do simulation of Flocking behavior on computer. His simulation program was known as "Boids" In this mathematical model each bird act as an independent agent communicating and cooperating with other neighboring agents. Objective is to measure from a realistic simulation the average energy spend by each bird, the angular momentum and the force that each bird has to withstand in a typical flight ritual.
\end{abstract}

\section{Introduction}
\label{sec:introduction}

Living beings are characterized by there behavior and appearance. The behavior of animals has been an important topic in the area of research. As with the use of technology we have been able to beat the animals in more and more areas by studying their behavior. If we go down to the level where creatures like ants, flocks, shoals, and swarms are, we can understand the mechanism of how these emerge and able to use the same mechanism to achieve what we have not achieved before. \\

With a simulation, we can understand flocking behavior. First, flocking behavior simulation was done by\textbf{ Craig Reynolds in 1986 named as "Boids"}.
The three rules formulated by Reynolds in his boids are still the basis of modern flocking simulation. Three rules of flocking are:-\\
\includegraphics[width=\textwidth]{694308f7-04ef-4eb0-b49e-640e3432f19c.jpg}\\

\textbf{Separation} - avoid crowding neighbors.

\textbf{Alignment} - steer towards average heading of neighbors.

\textbf{Cohesion} - steer towards average position of neighbors.\\

For simplicity and observing the behavior of boids, our model is bound to the simulation of flocking behavior on a 2D plane.

\section{Definitions}
\label{sec:theory}

\subsection{Boid}
A representation of a bird in flocking simulations. The word is most commonly used in the context of Boids, the original computer simulation of flocking behavior by
Reynolds and derivatives of Boids.

\subsection{Flocking Behavior}
In this model flocking behavior is used very lousily to mean any flock-like behavior, such as: Shoaling, swarming and maybe to some extent human crowds.

\subsection{Emergent behavior}
When individual objects interact with each other directly or indirectly to create much more complex results. Another characteristic of emergent behavior is that the end result is hard to predict even if each interaction in itself is simplistic. Ant hills, economics, the evolution are all examples of how agents abiding to relatively simple rules can create a system more complex than the sum of its parts. Reynolds
Boids algorithm can be said to create an emergent behavior.


\section{Reynolds Boid Model Algorithm}
Reynolds boid model follows simple structure: calculate the current state, draw the state on the screen and repeat. Algorithms given below shows the structure of typical boid implementation and the three rules of the boid algorithm which changes the speed and course of the boid. After the speed and course of boids are updated, we can update the position of the boids.

\begin{algorithm}
\begin{algorithmic}
\State \textbf{Data} : A group of boids.
\State \textbf{Result} : Simulates flocking behavior with an animation.
\For{\textbf{each} $Frame$ }
	\For{\textbf{each} $boid$}
    	\State separation(boid);
        \State cohesion(boid);
        \State alignment(boid);
    \EndFor
    
    \For{\textbf{each} $boid$}
    	 \State boid.x $\gets$ $cos(boid.course) * b.velocity * dTime;$
    	 \State boid.y $\gets$ $sin(boid.course) * b.velocity * dTime;$
         \State draw(boid);
    \EndFor
\EndFor 
\end{algorithmic}
\end{algorithm}

\subsection{\textbf{Cohesion}}
\begin{algorithm}
\begin{algorithmic}
\State \textbf{Data} : A boid.
\State \textbf{Result} : The course of the boid in updated.
 \State $goal \gets$ $(0,0);$
 \State $neighbours\gets$ getNeighbors(boid);
\For{\textbf{each} $nBoid$ $in$ $neighbours$}
         \State $goal \gets$ $goal$ $+$ $positionOf(nBoid)$   
        \EndFor 
           \State $goal \gets$ $goal$ $/$ $neighbours.size();$
              \State $steerForward(goal, boid);$ 
\end{algorithmic}
\end{algorithm}
 Cohesion is the rule that keeps the flock together, without it there would not be any flocking at all.It finds the average position of the neighborhood boids and tries to move the boid towards it.\\
 \\
 \\
 \\
\subsection{Separation}
\begin{algorithm}
\begin{algorithmic}
\State \textbf{Data} : A boid.
\State \textbf{Result} : The course of the boid in updated.
 \State $goal \gets$ $(0,0);$
 \State $neighbours\gets$ getNeighbors(boid);
\For{\textbf{each} $nBoid$ $in$ $neighbours$}
         \State $goal \gets$ $goal$ $+$ $positionOf(Boid)$ $-$ $positionOf(nBoid)$   
        \EndFor 
           \State $goal \gets$ $goal$ $/$ $neighbours.size();$
              \State $steerForward(goal, boid);$ 
\end{algorithmic}
\end{algorithm}

This rule attempts steer the boid away from possible collisions. It’s important to note that the distance from which the boids start to avoid each other must be less than the distance from which the boids attract each other.
\newpage
\subsection{Alignment}
\begin{algorithm}
\begin{algorithmic}
\State \textbf{Data} : A boid.
\State \textbf{Result} : The course and velocity of the boid is updated.
 \State $dCourse \gets$ 0;
 \State $dVelocity \gets$ 0;
 \State $neighbors \gets$ getNeighbors(boid);
\For{\textbf{each} $nBoid$ $in$ $neighbours$}
         \State $dCourse \gets$ $dCourse$ $+$ $getCourse(nBoid)$ $-$ $getCourse(boid)$
          \State $dVelocity \gets$ $dVelocity$ $+$ $getVelocity(nBoid)$ $-$ $getVelocity(boid)$
        \EndFor 
           \State $dCourse \gets$ $dCourse$ $/$ $neighbours.size();$
            \State $dVelocity \gets$ $dVelocity$ $/$ $neighbours.size();$
              \State $boid.addCourse(dCourse)$
              \State $boid.addVelocity(dVelocity)$
\end{algorithmic}
\end{algorithm}
This rule tries to make the boids mimic each others course and speed. If this rule was not used the boids would bounce around a lot and not form the beautiful flocking patterns that can be seen in real flocks.

\subsection{Neighborhood}
Core part of Reynolds formula is the neighborhood that boids perceives which decides about the next move according to the boids. Each algorithm of three rules uses a function called getNeighbour(Boid) to get the number of given boid.\\
\\
Another possible definition of the neighborhood is to let each boid look at the \textit{N} closest voids instead.
\section{Conclusion}
Reynolds formula was the one which with the realistic simulation of the boids helped to find  the average energy spend by each bird, the angular momentum and the force that each bird has to withstand in a typical flight ritual. Introduction of neighborhood does not change the classic rules of Reynolds formula. The N closest definition given an advantage that it allows a larger search radius.\\
\\
Since the number of neighbors is not limited by the radius but of N. If one choses
to use N closest it’s important to lower the weight of the alignment if the boid is
far away or else the boids will steer away from each other instead of forming a flock.
However if the all boids has formed flocks of sizes greater than N it becomes harder
for each individual flock to notice the other flocks. Since the boids in each flock
only sees the N closest boids within the flock, it won’t see the other flocks unless
they bump into each other.\\
\\

This is less of an issue with the within a radius definition, since it will always
notice boids within the radius. Thus I would recommend the within a certain radius
definition if the system contains many boids. But if the boids is separated by a great
distance the N closest would be a good choice since it’s good at finding new boids
until it’s in a flock of size greater than or equal to N.
\begin{thebibliography}{9}
\bibitem{nano3}
  Craig W. Reynolds. Flocks, herds, and schools: A distributed behavioral model.
In \textit{Computer Graphics}, pages 25–34, 1987.
\bibitem{nano3}
 Craig W. Reynolds. Boids, Background and Update . http://www.red3d.com/
cwr/boids/, 2007. [Online; accessed 14-April-2011].
\bibitem{nano3}
F. Heppner and U. Grenander. A stochastic nonlinear model for coordinated
bird flocks. 1990.
\bibitem{nano3}
$https://en.wikipedia.org/wiki/Flocking_(behavior).$
\bibitem{nano3}
$http://www.pnas.org/content/111/29/10422$
\end{thebibliography}
\end{document}